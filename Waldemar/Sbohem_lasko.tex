%!TEX ROOT=..\zpevnik tex.tex

1. Ať bylo [C]mně i [F]jí tak [G]\null šestnáct [C]let, [F G] 

zeleným [C]\null údolím [Am]jsem si ji [D7]ved',[G7] 

byla [C]krásná, to [C7]vím, a já měl st[F]rach, jak [Fm]\null říct,
když na řa[C]sách slzu [G7]má velkou jako h[C]rách. [F C]

R: [C7]Sbohem, [F]lásko, nech mě [Dm]jít, nech mě [Em]jít, bude [Am]klid,
žádnej [Dm]pláč už nespra[G7]ví ty mý [C]nohy toula[C7]vý,
já tě [F]vážně měl moc [Dm]rád, co ti [Em]víc můžu [Am]dát,
nejsem [Dm]\null žádnej id[G7]eál, nech mě [C]jít zas [F]o dům [C]dál.

2. A tak šel čas, a já se toulám dál,
v kolika údolích jsem takhle stál,
hledal slůvka, co jsou jak hojivej fáč,
bůhví, co jsem to zač, že přináším všem jenom pláč.

R:

Rec: Já nevím, kde se to v člověku bere - ten neklid, co ho tahá z místa na místo, co ho nenechá, aby byl sám se sebou spokojený jako většina ostatních, aby se usadil, aby dělal jenom to, co se má, a říkal jenom to, co se od něj čeká, já prostě nemůžu zůstat na jednom místě, nemůžu, opravdu, fakt.

R:

