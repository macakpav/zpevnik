%!TEX ROOT=..\zpevnik.tex

[Am]\null Šel včera městem [Em]muž 
a šel po hlavní [Am]třídě, [Em]

[Am]\null šel včera městem [Em]muž 
a já ho z okna [Am]viděl,
[C]na flétnu chorál [G]hrál, 
znělo to jako [Am]zvon
a byl v tom všechen [Em]\null žal, 
ten krásný dlouhý [F]tón,
a já jsem náhle [F#dim]věděl: 
Ano, to je [E7]on, to je [Am]on.

Vyběh' jsem do ulic jen v noční košili,
v odpadcích z popelnic krysy se honily
a v teplých postelích lásky i nelásky
tiše se vrtěly rodinné obrázky,
a já chtěl odpověď na svoje otázky, otázky.

[Am, Em, C, G, Am, F, F#dim, E7]\null®:Na na na, 2x
Dohnal jsem toho muže a chytl za kabát,
měl kabát z hadí kůže, šel z něho divný chlad,
a on se otočil, a oči plné vran,
a jizvy u očí, celý byl pobodán,
a já jsem náhle věděl kdo je onen pán onen pán

Celý se strachem chvěl, 
když jsem tak k němu došel,
a v ústech flétnu měl od Hieronyma Bosche,
stál měsíc nad domy jak čírka ve vodě,
jak moje svědomí, když zvrací v záchodě,
a já jsem náhle věděl: 
to je Darmoděj, můj Darmoděj.

®:[Am]Můj Darmoděj[Em], 
vagab[C]und osu[G]dů a lásek,
[Am]jenž prochází [F]všemi [F#m]sny, 
ale dnům [E7]vyhýbá se,
můj Darmoděj, krásné zlo, 
jed má pod jazykem,
když prodává po domech jehly se slovníkem.

Šel včera městem muž, podomní obchodník,
šel, ale nejde už, krev skápla na chodník,
já jeho flétnu vzal a zněla jako zvon
a byl v tom všechen žal, ten krásný dlouhý tón,
a já jsem náhle věděl: 
ano, já jsem on, já jsem on.

®:Váš Darmoděj, vagabund osudů a lásek,
jenž prochází všemi sny, ale dnům vyhýbá se,
váš Darmoděj krásné zlo jed mám pod jazykem
když prodávám po domech jehly se slovníkem.
bude to písnička o nás a kometě ...