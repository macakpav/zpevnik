%!TEX ROOT=..\zpevnik.tex

1. [C]Když jsem byl malý, říkali mi naši:
"Dobře se uč a jez chytrou kaši,
[F]až jednou vyrosteš, [G]budeš doktorem [C]práv.
Takový doktor sedí pěkně v suchu,
bere velký peníze a škrábe se v uchu,"
[F]já jim ale na to řek':"Ch[G]ci být hlídačem [C]krav."

R: Já chci [C]mít čapku s bambulí nahoře,
jíst kaštany a mýt se v lavoře,
[F]od rána po celý [G]den zpívat si [C]jen,
zpívat si: pam pam pam[F, G, C] ...

2. K vánocům mi kupovali hromady knih,
co jsem ale vědět chtěl, to nevyčet' jsem z nich:
nikde jsem se nedozvěděl, jak se hlídají krávy.
Ptal jsem se starších a ptal jsem se všech,
každý na mě hleděl jako na pytel blech,
každý se mě opatrně tázal na moje zdraví.

R:

3. Dnes už jsem starší a vím, co vím,
mnohé věci nemůžu a mnohé smím,
a když je mi velmi smutno, lehnu si do mokré trávy.
S nohama křížem a s rukama za hlavou
koukám nahoru na oblohu modravou,
kde se mezi mraky honí moje strakaté krávy.

R: