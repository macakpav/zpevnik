%!TEX ROOT=..\zpevnik tex.tex

[Emi]Ležel jsem [D]na posteli [C]venku [H]padal [Emi]sníh
[C]já zrovna [G]rozjímal o [C]věcech intim[H]ních
[C]dal jsem si v [G]rozechvění z [C]flašky piva [H]hlt
[Emi]když do kvar[D]týru pro mě [C]přišla [H]kmotra [Emi]smrt.

\begin{multicols}{2}
V obvyklém podobenství kostry člověčí
na dveře zaťukala v komoře zvenčí
potvora přihrbená s kosou v pařátách
pozvolna překročila mého hnízda práh.

Slušně se představila jsem smrt obecná
tisíce nebožtíků důvěrně mě zná
řekla mi nachystej se hochu už je čas
a po mých zádech na to lehce přejel mráz.

Jen tolik nespěchejte milá kmotřičko
támhle se posaďte sečkejte maličko
musím se na tu cestu řádně připravit
něco piv z basy vypít dýmku zapálit.

Doufám že pohoštění malé přijmete
a trošku chmelovinky neodmítnete
nechtěla zprvu já však nedal pokoje
až do lebky jí tekly chladné Prazdroje.

Po třetím pivu začla kmotra vyprávět
jak s kosou po zemi se vláčí tisíce let
plzeňský rozvázalo smrti čelisti
a o záhrobí vykládala mi zvěsti.

Nemusím prý se vůbec téhle cesty bát
říkala zubatá že mohu být jen rád
v záhrobí totiž vládne pouze klid a mír
povídá budeš se mít lépe nežs tu žil.

Nejsou tam žádný schůze na terejch jen spíš
z ničeho platit daně nikdy nemusíš
lumpy a práskače tam nikdy nespatříš
klid rukám dopřeješ však hochu uvidíš.

I když má každý člověk ze mě jenom strach
jsem vlastně svoboda a vůbec už ne vrah
dost ale řečí už co bych se chlubila
od pěny lebku mám však nejsem opilá.

A tak jsme vyšli kmotra zubatá a já
na dlouhou cestu která návratu nemá
jenom se nebojte i pro vás příjde si
i vám se lidé milí blýská na časy.
jenom se nebojte i pro vás příjde si
i vám se lidé milí blýská na časy.
\end{multicols}