%!TEX ROOT=../zpevnik.tex

1. [F]Jedenkrát dávno šel [Dmi]pustinou muž

a na [C]pravým místě tam [B]vykácel buš,

z klád postavil [C]dům, že měl [F]sílu jak [C]býk,

[B]rozoral zem jako [Gmi]válečník.



2. Jenže za ním jdou další, a ti umí víc,

vázaný krovy a zdi z vepřovic,

a do zlatejch polí a bučení krav

po nějaký době zní telegraf.

3. A už je tu kostel, a konečně most,

a železná ruda a zločinnost

a okresní město má okresní soud

a [B]ta stará trať jméno Telegraf [Dmi]Road.



4. Těžký jdou časy teď znova a znova,

skončila válka a chystá se nová,

a rozmoklou stezkou, co prošel ten skaut,

v [B]deseti proudech jdou pro[Gmi]vazy aut jak dr[B]avá [F]\null řeka.



*: A [Gmi]rádio hlásí, že v noci byl mráz,

[F]lidi jdou z práce a nemají čas,

[C]jenže vlak domů má [Ami]velký zpožděn[Dmi]\null í.



5. Už nemůžu dělat, co kde bych kdy chtěl,

třeba v pralesích kácet, to bohužel,

můžu jen sklízet, co zasil jsem sám,

a zaplatit všechno, co komu kde mám.

6. Ti šedaví ptáci na drátech z mědi

vo tomhletom kódu už ledacos vědí,

ti můžou letět a zapomenout

[B]na celej řád týhle Telegraf [Dmi]Road.



7. Víš, že bývaly časy, kdy bylo to zlý,

že spali jsme v dešti a promrzlí,

teď, když mi říkáš: tak vem si, co chceš,

moc dobře cítím, že už je to lež.



8. Tak důvěřuj ve mě, dej kočímu bič,

já vezmu tě s sebou a odvedu pryč

[B]ode všech [C]temnot a [F]vysokejch [C]zdí, [B]

od všeho strachu, [C]co v [F]ulicích spí[C, B]

já prošel jsem pamě[C]tí k[F]dekterej [C]kout[B]

a viděl jen smutek [C]jak [F]pýcha se [C]dmout[B]

a chtěl bych [F]zapome[C]nout

na [Ami]všechny ty zákazy vj[Dmi]ezdu, který jsou

rozse[B]tý po celý Telegraf [Dmi]Road ...
