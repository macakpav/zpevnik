%!TEX ROOT=..\zpevnik.tex

[C]Lidé o ní [Am]\null říkají,
[F]\null že je v lásce [G]nesvépravná.
A něco o tom [C]vím.

[C]Tu ženskou bych zabil, polil benzínem,
zapálil a nechal bych ji kouřit komínem,
[F]ale před tím bych ji mučil, dával bych mravence
[C]za podprdu a za trenky za ty její tendence.

Mě milovat tak, až mě to škrtí,
pejsek ocáskem totiž stále nevrtí.
A když by to i zkoušel, tak by mu upadl
a jak ta polní lilie by uvadl.

R:
A když se [G]podívá,
tak světlo v [Am]očích [F]má a já vím, [C]\null že
[G]na vině jsem [C]já.

[C]Lidé o ní [Am7]\null říkají…
\slpc
Přemýšlel jsem dokonce i o sebevraždě,
třeba by mě po ní začala brát konečně vážně,
třeba by mě nechala konečně vydechnout,
třeba bych se mohl pod zemí konečně nadechnout.

Miluje mě tak, že mně všichni závidí,
jenže ten zbytek ledovce už nikdo nevidí,
a ten mi dělá zimu a za těchto podmínek
se stále víc scvrkávám tak, jak můj šulínek.

Refrén
Lidé o ní říkají…

Pokoušel jsem vyhledat odbornou pomoc,
i odborná pomoc mi řekla: To je moc,
musíš věřit v slunce, teď užij si svou noc,
třeba vyjdou hvězdy alespoň ad hoc.

Pokoušel jsem myslet tedy na jinou,
našla si ji na facebooku, přejela ji šalinou,
takže všechny děvčata se mě teď bojí,
co ona spojí, to nikdo nerozpojí.

Refrén

Kamarád, co žije sám, by ji možná chtěl,
jenže nebere mi telefon, asi se dozvěděl,
jaké její krása má vedlejší účinky,
jaký mi to po těle dělá pupínky.
Na náhrobní kámen mi jednou dejte:
Teď užívá si nebe, protože žil v pekle,
a běda tomu, jenž by se neměl na pozoru,
tomu, jenž by poznal mou truchlící vdovu.

Refrén



