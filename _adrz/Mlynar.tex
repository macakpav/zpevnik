%!TEX ROOT=../zpevnik.tex


: Jede[G]n mlynář [Em]dceru měl, [C]každý mu ji [D]záviděl. : 
R:        Bajó, ba né, spíš jó než né, záviděl. 

: Dva bratři se smluvili, jak by mu ji zkurvili. : 

: Jeden si do pytle vlez, druhej ho do mlejna vez. : 

: Kam tatíčku mlynáři, kam ten pytel máme dát. : 

: Dejte ho do světničky, naší milé Aničky. :   

: A když přišlo klekání, pytel dostal cukání. : 

: A když vyšel měsíček, měla ho tam kousíček. : 

: A když vyšel Merkur, měla ho tam vejpůl. : 

: A když vyšli hvězdičky, měla ho tam celičký. : 

: A když vyšel Jupiter, měla ho tam po pytel. : 

: A když půlnoc odbyla, panna pannou nebyla. :

: A když bylo půl druhý, měla ho tam podruhy. : 

: A když bylo pul třetí, měla ho tam potřetí. : 

: Ona vzala dvě hůlky, praštila ho přes kulky. :

: A von na ní lopatou, přes tu kundu chlupatou. :

: A když přišlo svítání, pytel z mlejna uhání. : 

: Na svatýho Augusta, Anče pupek narůstá... :

: Na svatýho Řehoře, parchant běhá po dvoře... :

: A když bylo po roce, topil děcko v potoce. : 
