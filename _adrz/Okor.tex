%!TEX ROOT=..\zpevnik.tex

1. [C]Na Okoř je cesta jako žádná ze sta,
[G7]vroubená je stroma[C]ma.
[C]Když du po ní v létě samoten ve světě,
[G7]sotva pletu noha[C]ma. 
[F]Na konci té cesty [C]trnité
[D]stojí krčma jako [G7]hrad.
[C]Tam zapadlí trampi hladoví a sešlí,
[G7]začli sobě noto[C]vat.

R: [C]Na hradě Okoři [G7]světla už nehoří 
[C]bílá paní [G7]\,šla už dávno [C]spát 
[C]ona měla ve zvyku [G7]podle svého budíku 
[C]o půlnoci [G7]chodit straší [C]vat 

[F]od těch dob co jsou tam [C]trampové 
[D]nesmí z hradu [G7]pryč 
[C]a tak dole v podhradí [G7]se šerifem dovádí 
[C]on ji sebral [G7]od komnaty [C]klíč 
\columnbreak
2. Jednoho dne z rána roznesla se zpráva, 
že byl Okoř vykraden .
Nikdo neví dodnes, kdo to tenkrát odnes,
nikdo nebyl dopaden.

Šerif hrál celou noc mariáš 
s bílou paní v kostnici. 
Místo aby hlídal, zuřivě ji líbal,
dostal z toho zimnici.