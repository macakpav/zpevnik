%!TEX ROOT=..\zpevnik.tex

1.[E]Klidně mluvte sám, [G]já Vás když tak doplním, 
či [D]případně se něco poze[E]ptám.
Takže začít sám. Kde však jen začít mám?
Já začátek Vám možná přenechám.
Tak začněme hned, zda Vám to nevadilo, 
souhrnnou otázkou třicet pět.
Takže třicet pět. To je ta z těch na konci,
tam bojím se, že neznám odpověď.

[E]Hlavně klid, já [G]nevidím to marně,
[D]\nullčlověk občas musí bruslit, 
[A]ač je zrovna na plovárně,
zdárně ono to dopadne,
ač teď se to zdá nesnadné.

R: [E]Ne, ne já [G]netušim,
[D]učil jsem se, učil jsem se, [A]učil jsem se na mou duši,
[E]ne, ne já [G]nemám šajn,
[D]trojka byla by fakt [E]fajn.
Nananá… \slpc


2.Jste jen nervózní, já Vám zcela rozumím,
však já věřím, že to tam někde je. 
Třeba mi dozní kousek ňáký přednášky,
spíš je to ale fakt bez naděje. 
Tak co tenhle graf? Popište ho, a hned by se
Vám pátá otázka pokryla.  
Z tý jsem právě paf. Jinde věděl bych, však ta v tom
vyfoceným testu nebyla.
Co?
Co?

R:

[E]Co mám s Vámi dělat? [G]Tu zkoušku Vám nedat?
[D]Test se Vám však povedl, to [A]abych něco vyzvedl, jen
[E]Co je ale pravda, [G]máte na kahánku.
[D]Co byste si dal za známku?
[A]Co že bych si dal za známku?
[A]Co byste si dal za známku?
[A]Co že bych si dal za známku?

R: