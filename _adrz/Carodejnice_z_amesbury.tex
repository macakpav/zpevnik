%!TEX ROOT=..\zpevnik.tex

1. Zuzana [Am]byla dívka, [G]která žila v [Am]Amesbury,
s jasnýma [C]očima a [G]\null řečmi pánům [Am]navzdory,
souse[C]dé o ní [G]\null říkali, že [Am]temná kouzla [Em]zná
a [F]\null že se lidem [Em]vyhýbá a s [F]\null ďáblem [G]pletky [Am(D)]má.
\begin{multicols}{2}
	
2. Onoho léta náhle mor dobytek zachvátil
a pověrčivý lid se na pastora obrátil,
že znají tu moc nečistou, jež krávy zabíjí,
a odkud ta moc vychází, to každý dobře ví.

3. Tak Zuzanu hned před tribunál předvést nechali,
a když ji vedli městem, všichni kolem volali:
"Už konec je s tvým řáděním, už nám neuškodíš,
teď na své cestě poslední do pekla poletíš!"

4. Dosvědčil jeden sedlák, že zná její umění,
ďábelským kouzlem prý se v netopýra promění
a v noci nad krajinou létává pod černou oblohou,
sedlákům krávy zabíjí tou mocí čarovnou.

5. Jiný zas na kříž přísahal, že její kouzla zná,
v noci se v černou kočku mění dívka líbezná,
je třeba jednou provždy ukončit ďábelské řádění,
a všichni křičeli jako posedlí:"Na šibenici s ní!"
\slpc
6. Spektrální důkazy pečlivě byly zváženy,
pak z tribunálu povstal starý soudce vážený:
"Je přece v knize psáno: nenecháš čarodějnici žít
a před ďáblovým učením budeš se na pozoru mít!"

7. Zuzana stála krásná s hlavou hrdě vztyčenou
a její slova zněla klenbou s tichou ozvěnou:
"Pohrdám vámi, neznáte nic nežli samou lež a klam,
pro tvrdost vašich srdcí jen, jen pro ni umírám!"

8. Tak vzali Zuzanu na kopec pod šibenici
a všude kolem ní se sběhly davy běsnící,
a ona stála bezbranná, však s hlavou vztyčenou,
zemřela tiše samotná pod letní oblohou ...
\end{multicols}
